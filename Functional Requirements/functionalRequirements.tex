\documentclass[hidelinks, 12pt, oneside]{article}
\usepackage{bookmark}
\usepackage{graphicx}
\usepackage{hyperref}
\usepackage[utf8]{inputenc}
\usepackage[english]{babel}


\begin{document}

	\begin{center}
    \centering
    
%University logo
    \includegraphics[width=300px]{img/UP_Logo.png}
    \rule{0\linewidth}{0.15\linewidth}\par
    
    		\begin{center}
		{\uppercase{\Large Architectural Requirements\par}}
   		{\Large Mobile Monitoring App \par} 
   		{\Large Emilio Mumba  \par} 
    		\vspace{1cm}
		   		
    		{\Large The 5 Concurrent Nodes \par} 
    		\vspace{1cm}
		
		{\normalsize Khathutshelo Shaun Matidza\par}
		{\normalsize Sylvester Sandile Mpangane\par}
		{\normalsize Thabang Michael Letageng\par}
		{\normalsize Aluwani Augustine Simetsi\par}
		{\normalsize Matthew Nel\par}
		
		\end{center}

		\textbf{}		
		\centering
		\vspace{2cm}
		Department of Computer Science, University of Pretoria

		
	 	{\Large  May 2015}
\end{center}
\clearpage


	\tableofcontents
	\newpage
	\section{Vision and Scope}
	\subsection{Project Vision}
	The development of a mobile monitoring application by the computer science department would assist in mobile device security and proactive measures. Mobile devices comprise of a mass heterogeneous data sources. This personal data can be accessed by criminals through various malicious means such as the use of malware, applications, internet browsers, and social media. To protect people, there are law against accessing personal information without permission from an individual. An example of a privacy law in place is the Protection of Personal Information Act 4 of 2013 of South Africa. It is noted that there is minimal research towards proactive measures to protect personal data on mobile devices. Digital forensics is an emerging field that focuses on such research in a proactive (readiness) and reactive (investigative) manner. 
The proposal of a mobile monitoring application would promote readiness in digital forensics and protect mobile users from malicious entities and activities. Digital forensics is defined as the use of scientifically derived and proven methods towards the preservation, collection, validation, identification, analysis, interpretation and presentation of digital evidence derived from digital sources for the sole purpose of facilitating or furthering the reconstruction of events found to be criminal or helping to anticipate the unauthorized actions shown to be disruptive to planned operations [3].
Readiness is considered as the process of being prepared for a digital investigation before an incident has occurred [2]. The mobile monitoring application aims to provide proactive measure that are undertaking by the mobile device user or mobile device owner. Having this application installed on mobile devices will proactively ensure that relevant digital evidence is made ready available before an incident occurs. The mobile monitoring application is expected to monitor user activities on a mobile device and report application data/logs to a dashboard on a desktop computer. The mobile monitoring application generates reports giving the investigator quick and comprehensive data/logs that provide a starting point during a mobile device investigation. 
The objective of the mobile monitoring application is to display and assist in understanding the activities performed by a mobile user as well as shedding more light into the behaviour of the mobile user. Combining activities from the various applications promotes a proactive approach which in turn enforces proactive (readiness) measures. 
	\subsection{Project Scope}
		{\centering
		
		%\includegraphics[width=300px]{img/}
		\vspace{1cm}
		Figure 1: High level system modules
		}
	
	\section{Application Requirements and Design}
	\subsection{MobileMonitoringApp - Accounts module}
	
	\subsection{MobileMonitoringApp - Data module}
	
	\subsection{MobileMonitoringApp - Reports module}
		
\end{document}